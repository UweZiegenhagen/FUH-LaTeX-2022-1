\documentclass[12pt,ngerman,parskip=half]{scrreprt}

\usepackage{babel}
\usepackage{blindtext}

\usepackage{paralist}
\usepackage{microtype}
\usepackage{xcolor}
\usepackage{graphicx}
\usepackage{booktabs}

\definecolor{FUH}{RGB}{150,100,100}
\definecolor{fuh}{rgb}{0.5,0.1,0.5}

\author{Donald Duck}
\title{Hallo Fernuni Hagen}
\date{Köln, den 23.06.2022}

\newcommand{\physiker}[1]{\textcolor{FUH}{\textsc{#1}}}

\newcommand{\Physiker}[2]{\textcolor{fuh}{\textbf{#1} \textsc{#2}}}


\begin{document}
\maketitle

\tableofcontents

\listoffigures

\listoftables

\chapter{Hallo Welt}

Siehe Abbildung \ref{fig:katze} auf Seite \pageref{fig:katze}, siehe auch Abschnitt \ref{sec:hauptteil}

Siehe Abbildung \ref{fig:katze2} auf Seite \pageref{fig:katze2}.

\section{Einleitung}
\subsection{Das Thema an und für sich}
\subsubsection{Literatur}

\blindtext[2] 

\subsubsection{Stand der Forschung}

\blindtext[1] 


\begin{itemize} % Standard-LaTeX
\item Hallo
\item Fernuni
\item Hagen
\begin{itemize}[+]
\item Hallo
\item Fernuni
\item Hagen
\item Welt
\end{itemize}
\item Welt
\end{itemize}

\begin{compactitem}[\%] % Paket paralist
\item Hallo
\item Fernuni
\item Hagen
\begin{compactitem}[+]
\item Hallo
\item Fernuni
\item Hagen
\item Welt
\end{compactitem}
\item Welt
\end{compactitem}


\begin{enumerate}
\item Hallo
\item Fernuni
\item Hagen
\begin{enumerate}
\item Hallo
\item Fernuni
\item Hagen
\item Welt
\end{enumerate}
\item Welt
\end{enumerate}

\begin{compactenum}[I]
\item Hallo
\item Fernuni
\item Hagen
\begin{compactenum}[A.]
\item Hallo
\item Fernuni
\item Hagen
\item Welt
\end{compactenum}
\item Welt
\end{compactenum}


\begin{description}
\item[Apfel] eine Frucht vom Baum\footnote{\blindtext}
\item[Birne] auch eine Frucht vom Baum

\begin{description}
\item[Apfel] eine Frucht vom Baum
\item[Birne] auch eine Frucht vom Baum
\item[Tomate] ein Gemüse vom Strauch
\end{description}

\item[Tomate] ein Gemüse vom Strauch
\end{description}


\section{Hauptteil}\label{sec:hauptteil}

\blindtext[1] 

\blindtext[1]

\section{Fazit}

\blindtext[1] 

\chapter{Irgendwas}

Hallo, ich bin ein \textbf{fettgedrucktes} Wort. Ich bin ein \textit{kursiv} gesetztes Wort. Und ich bin sowohl \textbf{\textit{fett und kursiv}} gesetzt. Ich bin ein \textsl{geneigtes}, aber nicht \textit{kursives} Wort. \textsc{Kapitälchen} nutzt man auch gelegentlich. \physiker{Albert Einstein} war ein berühmter Physiker.

\Physiker{Marie}{Curie}

\includegraphics[width=2cm]{./Bilder/Katze} % Formate jpg, png, pdf


\chapter{Bilder einbetten}

\blindtext

\begin{figure}[h] % h = here, t = top, b = bottom, p = page
\begin{center} % \usepackage{here} => Option 'H' für schärferes h
\includegraphics[width=0.9\textwidth]{./Bilder/Katze.jpg} % Formate jpg, png, pdf
\caption{Melli, die Katz}\label{fig:katze}
\end{center}
\end{figure}

\blindtext[2] fssd fsdsdfsdf s sdfsdfsdf fssd fsdsdfsdf s sdfsdfsdf fssd fsdsdfsdf s sdfsdfsdf fssd fsdsdfsdf s sdfsdfsdf fssd fsdsdfsdf s sdfsdfsdf fssd fsdsdfsdf s sdfsdfsdf fssd fsdsdfsdf s sdfsdfsdf fssd fsdsdfsdf s sdfsdfsdf fssd fsdsdfsdf s sdfsdfsdf fssd fsdsdfsdf s sdfsdfsdf 


\begin{minipage}{\textwidth}
\begin{center}
\includegraphics[width=0.9\textwidth]{./Bilder/Katze.jpg} % Formate jpg, png, pdf
\captionof{figure}{Melli, die Miezekatze}\label{fig:katze2}
\end{center}
\end{minipage}


\chapter{Tabellen}\label{ch:tabelle}

\begin{tabular}{|l|c|r|p{5cm}|} \hline
\textbf{Spalte 1} & \textbf{Spalte 2} & \textbf{Spalte 3} & \textbf{Spalte 4} \\ \hline
123 & 121 & 435345241 & Hallo, ich bin ein Satz, der nach fünf Zentimetern umgebrochen wird. \\ \hline
12322 & 12421 & 43241 & Hallo, ich bin ein Satz \\ \hline
\end{tabular}


\begin{table}[h]
\begin{center}
\caption{Eine Tabellenüberschrift}
\begin{tabular}{lcrp{5cm}} \toprule
\textbf{Spalte 1} & \textbf{Spalte 2} & \textbf{Spalte 3} & \textbf{Spalte 4} \\ \midrule
123 & 121 & 435345241 & Hallo, ich bin ein Satz, der nach fünf Zentimetern umgebrochen wird. \\ 
12322 & 12421 & 43241 & Hallo, ich bin ein Satz \\ \bottomrule
\end{tabular}
\end{center}
\end{table}

Hallo, ich bin ein Umbruch \\ in einer Zeile

Mit dem clearpage Befehl kann man eine Seite beenden

\clearpage

Hallo, ich bin kein Umbruch, 

sondern zwei Absätze 




\end{document}



\begin{center}
\includegraphics[height=0.25\textheight,angle=45]{./Bilder/Katze} % Formate jpg, png, pdf
\end{center}
