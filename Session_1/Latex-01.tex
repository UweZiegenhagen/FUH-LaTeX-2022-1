\documentclass[12pt,ngerman,parskip=half]{scrreprt}

\usepackage{babel}
\usepackage{blindtext}

\usepackage{paralist}

\usepackage{microtype}

\author{Donald Duck}
\title{Hallo Fernuni Hagen}
\date{Köln, den 23.06.2022}


\begin{document}
\maketitle

\tableofcontents

\chapter{Hallo Welt}

\section{Einleitung}
\subsection{Das Thema an und für sich}
\subsubsection{Literatur}

\blindtext[1] 

\subsubsection{Stand der Forschung}

\blindtext[1] 


\begin{itemize} % Standard-LaTeX
\item Hallo
\item Fernuni
\item Hagen
\begin{itemize}[+]
\item Hallo
\item Fernuni
\item Hagen
\item Welt
\end{itemize}
\item Welt
\end{itemize}

\begin{compactitem}[\%] % Paket paralist
\item Hallo
\item Fernuni
\item Hagen
\begin{compactitem}[+]
\item Hallo
\item Fernuni
\item Hagen
\item Welt
\end{compactitem}
\item Welt
\end{compactitem}


\begin{enumerate}
\item Hallo
\item Fernuni
\item Hagen
\begin{enumerate}
\item Hallo
\item Fernuni
\item Hagen
\item Welt
\end{enumerate}
\item Welt
\end{enumerate}

\begin{compactenum}[I]
\item Hallo
\item Fernuni
\item Hagen
\begin{compactenum}[A.]
\item Hallo
\item Fernuni
\item Hagen
\item Welt
\end{compactenum}
\item Welt
\end{compactenum}


\begin{description}
\item[Apfel] eine Frucht vom Baum
\item[Birne] auch eine Frucht vom Baum

\begin{description}
\item[Apfel] eine Frucht vom Baum
\item[Birne] auch eine Frucht vom Baum
\item[Tomate] ein Gemüse vom Strauch
\end{description}

\item[Tomate] ein Gemüse vom Strauch
\end{description}


\section{Hauptteil}

\blindtext[1] 

\blindtext[1]

\section{Fazit}

\blindtext[1] 

\chapter{Irgendwas}

Hallo, ich bin ein \textbf{fettgedrucktes} Wort. Ich bin ein \textit{kursiv} gesetztes Wort. Und ich bin sowohl \textbf{\textit{fett und kursiv}} gesetzt. Ich bin ein \textsl{geneigtes}, aber nicht \textit{kursives} Wort. \textsc{Kapitälchen} nutzt man auch gelegentlich.


\end{document}

