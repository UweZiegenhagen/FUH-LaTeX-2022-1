\documentclass[ngerman]{beamer}

\usepackage[]{babel}
\usepackage[]{blindtext}

\usetheme{PaloAlto}

\title{Meine erste Präsentation}
\author{Uwe Ziegenhagen}
\date{Köln, den \today}
\institute{Dante e.V.}

\setbeamerfont{caption}{size=\footnotesize}

\begin{document}

\begin{frame}

\maketitle

\end{frame}

\begin{frame}
\frametitle{Inhalt}

\tableofcontents

\end{frame}



\section{Einleitung}

\begin{frame}
\frametitle{Titel der Folie}
\framesubtitle{Untertitel der Folie}

\begin{itemize}
\item 
\item 
\item 
\end{itemize}

\begin{alert}{Hinweise}
{\scriptsize \blindtext}
\end{alert}


\end{frame}


\begin{frame}
\frametitle{Titel der Folie}
\framesubtitle{Untertitel der Folie}

\begin{enumerate}
\item 
\item 
\item 
\item 
\item 
\end{enumerate}


\end{frame}

\begin{frame}
\frametitle{Titel der Folie}
\framesubtitle{Untertitel der Folie}

\blindtext


\end{frame}

\begin{frame}
\frametitle{Titel der Folie}
\framesubtitle{Untertitel der Folie}

\begin{equation}
-\frac{p}{2} \pm 
\sqrt{ 
 \left(
    \frac{p}{2}
 \right)^2 -q 
}
\end{equation}

\end{frame}


\begin{frame}
\frametitle{Titel der Folie}
\framesubtitle{Untertitel der Folie}

\begin{figure}
\begin{center}
\includegraphics[width=0.7\textwidth]{./Bilder/Katze.jpg}
\caption{Schneeteufel}
\end{center}
\end{figure}

\end{frame}


\begin{frame}
\frametitle{Titel der Folie}
\framesubtitle{Untertitel der Folie}

\begin{columns}
\begin{column}{0.495\textwidth}
\begin{itemize}
\item a
\item b
\item c
\item d
\item e
\end{itemize}
\end{column}
\begin{column}{0.495\textwidth}
\begin{itemize}
\item f
\item g
\item h
\item i
\item j
\end{itemize}
\end{column}
\end{columns}

\end{frame}


\begin{frame}
\frametitle{Titel der Folie}
\framesubtitle{Untertitel der Folie}

\begin{enumerate}
\item<1-> abc
\item<2> def
\item<-3> ghi
\item<4-> jkl
\item<5-> mno
\end{enumerate}


\end{frame}


\end{document}