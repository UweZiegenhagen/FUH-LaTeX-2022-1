%!TEX TS-program = Arara
% arara: pdflatex: {shell: yes}
\documentclass[12pt,ngerman,parskip=half]{scrartcl}

\usepackage[utf8]{inputenc}
\usepackage[T1]{fontenc}
\usepackage{booktabs}
\usepackage{babel}
\usepackage{graphicx}
\usepackage{csquotes}
\usepackage{paralist}

\usepackage{xcolor}
\usepackage{listings}

\definecolor{colBack}{rgb}{1,1,0.9}
\definecolor{colKeys}{rgb}{0,0,1}
\definecolor{colIdentifier}{rgb}{0,0,0}
\definecolor{colComments}{rgb}{1,0,0}
\definecolor{colString}{rgb}{0,0.5,0}

\lstset{literate=%
    {Ö}{{\"O}}1
    {Ä}{{\"A}}1
    {Ü}{{\"U}}1
    {ß}{{\ss}}1
    {ü}{{\"u}}1
    {ä}{{\"a}}1
    {ö}{{\"o}}1
    {~}{{\textasciitilde}}1
}


\lstset{%
    float=hbp,%
    basicstyle=\ttfamily\footnotesize, %
    identifierstyle=\color{colIdentifier}, %
    keywordstyle=\color{colKeys}, %
    stringstyle=\color{colString}, %
    commentstyle=\color{colComments}, %
    columns=flexible, %
    tabsize=2, %
    frame=single, %
    extendedchars=true, %
    showspaces=false, %
    showstringspaces=false, %
    numbers=left, %
    numberstyle=\tiny, %
    breaklines=true, %
    backgroundcolor=\color{colBack}, %
    breakautoindent=true, %
    captionpos=b,%
    language={Python},
    morekeywords={copyfile, write, unlink}
}




\begin{document}

Hallo FUH

\begin{lstlisting}
import timeit
start = timeit.default_timer()

files = ['f:/willich.txt', 'f:/willich.txt',  'f:/willich.txt']
output = 'F:/kombiniert.txt'

filecount = len(files)
print(f'Processing {filecount} files')
 
with open(output, 'w') as outputfile:
    for counter, file in enumerate(files):
        print(counter, file)
        with open(file, 'r') as fin:
            data = fin.read().splitlines(True)

        if counter == 0: # first file, take everything except last line
            outputfile.writelines(data[:-1])
        elif counter == filecount - 1: # last file, take everything except first two lines
            outputfile.writelines(data[2:])

        else: # other files, everything except first two and last line
            outputfile.writelines(data[2:-1])

stop = timeit.default_timer()
print(f'Time: {round(stop - start, 1)} seconds')  
\end{lstlisting}

\clearpage

\begin{lstlisting}[language={Python},morekeywords={default_timer,read,splitlines,writelines}]
import timeit
start = timeit.default_timer()

files = ['f:/willich.txt', 'f:/willich.txt',  'f:/willich.txt']
output = 'F:/kombiniert.txt'

filecount = len(files)
print(f'Processing {filecount} files')
 
with open(output, 'w') as outputfile:
    for counter, file in enumerate(files):
        print(counter, file)
        with open(file, 'r') as fin:
            data = fin.read().splitlines(True)

        if counter == 0: # first file, take everything except last line
            outputfile.writelines(data[:-1])
        elif counter == filecount - 1: # last file, take everything except first two lines
            outputfile.writelines(data[2:])

        else: # other files, everything except first two and last line
            outputfile.writelines(data[2:-1])

stop = timeit.default_timer()
print(f'Time: {round(stop - start, 1)} seconds')  
\end{lstlisting}

Wenn ich einzelne Befehle im Fließtext highlighten möchte so kann ich dies tun: \lstinline[language={Python}]{print()}

\clearpage

\lstinputlisting[language={Python},linerange={5-10}]{test.py}

\end{document}