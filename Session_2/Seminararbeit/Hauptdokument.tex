%!TEX TS-program = Arara
% arara: pdflatex: {shell: yes}
% arara: biber
% arara: pdflatex: {shell: yes}
\documentclass[12pt,ngerman,parskip=half]{scrbook}

\usepackage{babel}
\usepackage{blindtext}
\usepackage{microtype}
\usepackage{csquotes}
\usepackage{graphicx} % nicht mehr graphics nutzen!
\usepackage{subcaption}

\usepackage{scrlayer-scrpage}
\pagestyle{scrheadings}

\ohead[\headmark]{\headmark} % outer head
\ofoot[\pagemark]{\pagemark} % outer foot
\ifoot[]{} % inner foot
\ihead[]{} % inner head
\cfoot[]{} % center foot
\chead[]{} % center head

\usepackage{hyperref}
\hypersetup{
    bookmarks=true,                     % show bookmarks bar
    unicode=false,                      % non - Latin characters in Acrobat’s bookmarks
    pdftoolbar=true,                        % show Acrobat’s toolbar
    pdfmenubar=true,                        % show Acrobat’s menu
    pdffitwindow=false,                 % window fit to page when opened
    pdfstartview={FitH},                    % fits the width of the page to the window
    pdftitle={My title},                        % title
    pdfauthor={Author},                 % author
    pdfsubject={Subject},                   % subject of the document
    pdfcreator={Creator},                   % creator of the document
    pdfproducer={Producer},             % producer of the document
    pdfkeywords={keyword1, key2, key3},   % list of keywords
    pdfnewwindow=true,                  % links in new window
    colorlinks=true,                        % false: boxed links; true: colored links
    linkcolor=blue,                          % color of internal links
    filecolor=blue,                     % color of file links
    citecolor=blue,                     % color of file links
    urlcolor=blue                        % color of external links
}	


%\includeonly{Kapitel03} % berücksichtige nur Kapitel03, lasse aber die Seitenzahlen so, als ob alle Kapitel inkludiert sind.

\usepackage[style=verbose,backend=biber]{biblatex}
\addbibresource{Literaturverzeichnis.bib} % authortitle-icomp

\begin{document}

\begin{titlepage}
{\large\bfseries Fernuniversität Hagen \\
Mathematisch-Naturwissenschaftliche Fakultät \\
Lehrstuhl für Informatik}

\vspace*{5cm}
\begin{center}
{\LARGE\bfseries\enquote{Verteilte Systeme im Zeitablauf}}
\end{center}

\vspace*{1cm}
\begin{center}
{\Large\bfseries Bachelorarbeit}
\end{center}


\begin{center}
{\large\bfseries  vorgelegt von }
\end{center}

\begin{center}
{\Large\bfseries Max Mustermann}
\end{center}


\vfill
\begin{tabular}{ll}
Erstgutachter: & Prof. Dr. Daniel Düsentrieb \\
Zweitgutachterin: & Prof. Dr. Marie Curie \\
\end{tabular}

\hfill Hagen, den \today
\end{titlepage}



\tableofcontents

\listoffigures

\chapter{fsdfgsd}

\cite{knuth:1984}

\cite{ziegenhagen:2017}

\parencite{knuth:1984}

\cite{ziegenhagen:2017}

\cite{Bohmann:2017}

Das hat Knuth\footcite{knuth:1984} in seinem Buch beschrieben.

Im Jahr \citeyear{knuth:1984} hat \citeauthor{knuth:1984} in seinem Werk \citetitle{knuth:1984} beschrieben, was \TeX\ ist.


%!TeX root=Hauptdokument.tex
\chapter{Einleitung}

\section{Literatur}
\subsection{Literatur vor 1900}


\subsubsection{Europa}

\blindtext 

\blindtext

\blindtext

\subsubsection{Asien}

\blindtext 

\blindtext

\blindtext

\subsubsection{Amerika}

\blindtext 

\blindtext

\blindtext

\subsection{Literatur nach 1900}

\subsubsection{Europa}

\blindtext 

\blindtext

\blindtext

\subsubsection{Asien}

\blindtext 

\blindtext

\blindtext

\subsubsection{Amerika}

\blindtext 

\blindtext

\blindtext


\section{Stand der Forschung}

\blindtext 

\blindtext

\blindtext

\include{Kapitel02}

%!TeX root=Hauptdokument.tex

\chapter{Fazit}

\section{Literatur}
\subsection{Literatur vor 1900}


\subsubsection{Europa}

\blindtext 

\blindtext

\blindtext

\subsubsection{Asien}

\blindtext 

\blindtext

\blindtext

\subsubsection{Amerika}

\blindtext 

\blindtext

\blindtext

\subsection{Literatur nach 1900}

\subsubsection{Europa}

\blindtext 

\blindtext

\blindtext

\subsubsection{Asien}

\blindtext 

\blindtext

\blindtext

\subsubsection{Amerika}

\blindtext[3]

\blindtext[3]

\blindtext[3]


\section{Stand der Forschung}

\blindtext [100]

\blindtext

\blindtext


\blindtext[100]

\printbibliography

\printbibliography[title={Artikel},type=article]

\printbibliography[title={Bücher},type=book]




\end{document}