\documentclass[12pt,ngerman]{scrreprt}

\usepackage[utf8]{inputenc}
\usepackage[T1]{fontenc}
\usepackage{booktabs}
\usepackage{babel}
\usepackage{graphicx}
\usepackage{csquotes}
\usepackage{paralist}
\usepackage{xcolor}
\usepackage{amsmath}

%\newcounter{formel}

\newcommand{\foobar}{\refstepcounter{equation}\theequation}

\newtheorem{theorem}{Theorem}[chapter]
\newtheorem{corollary}{Corollary}[chapter]
\newtheorem{lemma}{Lemma}[chapter]

\begin{document}

\chapter{aaa}
\section{fsdfs}

\begin{equation*}
a=b
\end{equation*}

\begin{equation}\label{xx} % Split von überlangen Formeln
\begin{split}
a& =b+c-d\\
& \quad +e-f\\
& =g+h\\
& =i
\end{split}
\end{equation}

\begin{multline}
\framebox[.65\columnwidth]{A}\\
\framebox[.5\columnwidth]{B}\\
\shoveright{\framebox[.55\columnwidth]{C}}\\
\framebox[.65\columnwidth]{D}
\end{multline}


\begin{equation}\label{e:barwq}\begin{split}
H_c&=\frac{1}{2n} \sum^n_{l=0}(-1)^{l}(n-{l})^{p-2}
\sum_{l _1+\dots+ l _p=l}\prod^p_{i=1} \binom{n_i}{l _i}\\
&\quad\cdot[(n-l )-(n_i-l _i)]^{n_i-l _i}\cdot
\Bigl[(n-l )^2-\sum^p_{j=1}(n_i-l _i)^2\Bigr].
\end{split}\end{equation}

%pmatrix, bmatrix, Bmatrix, vmatrix and Vmatrix 
\begin{equation}
\begin{pmatrix}
1 & 2 & 3\\
a & b & c
\end{pmatrix}
\end{equation}

\begin{equation}
\begin{bmatrix}
1 & 2 & 3\\
a & b & c
\end{bmatrix}
\end{equation}

\begin{equation}
\begin{Bmatrix}
1 & 2 & 3\\
a & b & c
\end{Bmatrix}
\end{equation}

\begin{equation}
\begin{vmatrix}
1 & 2 & 3\\
a & b & c
\end{vmatrix}
\end{equation}

\begin{equation}
\begin{Vmatrix}
1 & -2 & 3\\
a & b & -c
\end{Vmatrix}
\end{equation}

\( A \cap B  \cup C \Rightarrow \overline{A \cap B  \cup C} \in \Omega \)


\[f(x) = \begin{cases}
5 & x \geq 0 \\
23 & \, \text{sonst}
\end{cases}\]



\[ \overbrace{a^2 + b^2}^\text{Satz \foobar} = \underbrace{c^2 + d^2}_{\text{Beweis \foobar}} \]


\begin{theorem}
Let \(f\) be a function whose derivative exists in every point, then \(f\) 
is a continuous function.
\end{theorem}


\begin{theorem}
Let \(f\) be a function whose derivative exists in every point, then \(f\) is 
a continuous function.
\end{theorem}

\begin{theorem}[Pythagorean theorem]
\label{pythagorean}
This is a theorem about right triangles and can be summarised in the next 
equation 
\[ x^2 + y^2 = z^2 \]
\end{theorem}

And a consequence of theorem \ref{pythagorean} is the statement in the next 
corollary.

\section{dsfdsgdfgdfg}

\begin{corollary}
There's no right rectangle whose sides measure 3cm, 4cm, and 6cm.
\end{corollary}

You can reference theorems such as \ref{pythagorean} when a label is assigned.

\begin{lemma}
Given two line segments whose lengths are \(a\) and \(b\) respectively there is a 
real number \(r\) such that \(b=ra\).
\end{lemma}

\begin{theorem}
Let \(f\) be a function whose derivative exists in every point, then \(f\) is 
a continuous function.
\end{theorem}

\end{document}