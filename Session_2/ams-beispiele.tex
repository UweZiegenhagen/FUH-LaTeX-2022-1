\documentclass[12pt,ngerman]{scrartcl}

\usepackage[utf8]{inputenc}
\usepackage[T1]{fontenc}
\usepackage{booktabs}
\usepackage{babel}
\usepackage{graphicx}
\usepackage{csquotes}
\usepackage{paralist}
\usepackage{xcolor}
\usepackage{amsmath}

\begin{document}

\begin{equation*}
a=b
\end{equation*}

\begin{equation}\label{xx} % Split von überlangen Formeln
\begin{split}
a& =b+c-d\\
& \quad +e-f\\
& =g+h\\
& =i
\end{split}
\end{equation}

\begin{multline}
\framebox[.65\columnwidth]{A}\\
\framebox[.5\columnwidth]{B}\\
\shoveright{\framebox[.55\columnwidth]{C}}\\
\framebox[.65\columnwidth]{D}
\end{multline}


\begin{equation}\label{e:barwq}\begin{split}
H_c&=\frac{1}{2n} \sum^n_{l=0}(-1)^{l}(n-{l})^{p-2}
\sum_{l _1+\dots+ l _p=l}\prod^p_{i=1} \binom{n_i}{l _i}\\
&\quad\cdot[(n-l )-(n_i-l _i)]^{n_i-l _i}\cdot
\Bigl[(n-l )^2-\sum^p_{j=1}(n_i-l _i)^2\Bigr].
\end{split}\end{equation}

%pmatrix, bmatrix, Bmatrix, vmatrix and Vmatrix 
\begin{equation}
\begin{pmatrix}
1 & 2 & 3\\
a & b & c
\end{pmatrix}
\end{equation}

\begin{equation}
\begin{bmatrix}
1 & 2 & 3\\
a & b & c
\end{bmatrix}
\end{equation}

\begin{equation}
\begin{Bmatrix}
1 & 2 & 3\\
a & b & c
\end{Bmatrix}
\end{equation}

\begin{equation}
\begin{vmatrix}
1 & 2 & 3\\
a & b & c
\end{vmatrix}
\end{equation}

\begin{equation}
\begin{Vmatrix}
1 & -2 & 3\\
a & b & -c
\end{Vmatrix}
\end{equation}

\( A \cap B  \cup C \Rightarrow \overline{A \cap B  \cup C} \in \Omega \)

\end{document}