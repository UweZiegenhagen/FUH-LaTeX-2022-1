\documentclass[ngerman]{scrartcl}

\usepackage[T1]{fontenc}
\usepackage{booktabs}
\usepackage{babel}
\usepackage{graphicx}
\usepackage{csquotes}
\usepackage{paralist}
\usepackage{xcolor}
\usepackage{siunitx}
\usepackage{amsmath}

\sisetup{
list-final-separator =  { sowie },  %{ \GetTranslation{and} },
list-pair-separator = { \GetTranslation{and} },
range-phrase = { \GetTranslation{to (numerical range)} },
}

\newcommand{\sirange}[3]{\SI{#1}{#3} -- \SI{#2}{#3}}


\begin{document}


Hallo, ich bin eine Zahl \num{3,1415927}, ich auch \num{3,141e10}

Hallo, ich bin eine Einheit \si{m^2}, \si{\kilo\meter^2}, \si{\nano\meter\per\second^2}

\SI{106}{m^2} % <= This is the way

\(106 \text{m}^2 \not= 106 m^2  \) % so bitte nicht

\ang{90} % für Winkelangaben

\SI{35}{\celsius} % Für Gradangaben mit Temperatur Einheit

\SI{35}{\kelvin} % Für Gradangaben mit Temperatur Einheit


\begin{equation}
\SI{35}{\kelvin} \not= \SI{35}{\celsius}
\end{equation}

\SIrange{10}{20}{m^2}

\sirange{10}{20}{m^2}

10:00 - 12:00 Uhr % falsch

10:00 -- 12:00 Uhr % richtig

10:00 --- 12:00 Uhr % Keine Ahnung, ob auch in Dt. gebräuchlich

\numrange{10}{20}

\numlist{10;20;30;40}

\qty[per-mode= fraction]{3,1415927}{\meter\per\second}

\SI[per-mode= fraction]{3,1415927}{\meter\per\second}


\end{document}
